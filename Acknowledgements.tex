%Acknowledgments

\renewcommand{\baselinestretch}{2}
\small\normalsize
\hbox{\ }

\vspace{-.65in}

\begin{center}
\large{Acknowledgments}
\end{center}

\vspace{1ex}


I want to acknowledge my committee, the mentors I have had over the years, colleagues at University of Maryland, NIST, and most of all my family.
First I want to thank my parents for encouraging curiosity, letting me find my own way through life, and never ending love.
When I was five years old, I asked my father what someone did to earn a doctorate.
His response was simple, people get Ph.D. for making new discoveries.
I am sure either of us at the time thought someday I would be getting a Ph.D.
I did not take a direct route from high school to a Ph.D., a path that included count less concerts, cross-country road trips, marriage, unexpected moves, position at a Government agency I never heard of and having a son. All of which has made this experience richer.
I also want to acknowledge my Grandpa.
You are a role model and constant source of inspiration.
My committee, experts in diverse fields from computer science, applied mathematics, microbiology, community ecology, and metrology.
Your diverse expertise has made this work possible.
Hector asking the next questions, digging deeper into the data, but also encouaging me to push the project out to the world when I may not be ready. 
Mihai helping a biologist learn about string and network algorithms. 
Mihai and Hector recognised the value in understanding a measurement process and characterizing sources of error. 
Dr. Misaki Takabayshi my undergraduate research advisor for introducing me to research and the dark art of PCR.
Dr. Jayne Morrow, encouraging me to go back to school for my Ph.D. and continual encouragement along the way, "It would do the world a disservice not to get a Ph.D."
Prachi editing, working on wastewater project.
Joe getting me started with 16S data analysis and helpful conversations. 
Justin for walk breaks, chats about science, and life.
Corrada-Bravo and Pop labs great group of smart and nice people, it was a pleasure to work with.
NIST for supporting me pursuing my degree.
NIST statistician Steve Lund and Jim Filliben for first exposing me to exploratory data analysis and helping shape how to think about and approach data analysis.
Pop-tarts and coffee for the sugar and caffine boost needed to write just a little bit more.
Most of all I want to thank my wife Erica and son Charlie, helping pushing me through the hard times, with unconditional love and encouragement.
Erica telling me things I do not want to hear when I need to hear them most.
Charlie, thank you for letting me read microbiome papers to you and listening to me as I worked though roadblocks in my research.
More importantly having an infectious laugh that made it easy to take a break and forget about work for a little while.
"Cause it's bathnight, it's bathnight!"
