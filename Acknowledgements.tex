%Acknowledgments

\renewcommand{\baselinestretch}{2}
\small\normalsize
\hbox{\ }

\vspace{-.65in}

\begin{center}
\large{Acknowledgments}
\end{center}

\vspace{1ex}


This work was funded by the National Institute of Standards and Technology and an NIH R01HG005220 grant awarded to Dr. Héctor Corrada Bravo.
I want to thank my committee, Héctor Corrada Bravo, Mihai Pop, Marc Salit, O. Colin Stine, and Nathan Swenson.  

My committee included experts in computer science, applied mathematics, microbiology, community ecology, and metrology.
Their diverse areas of expertise helped make this work possible.
I especially want to thank my advisor, Héctor, for teaching me to always ask the next question, dig deeper into the data, and also for encouaging me to push the project out into the world. 
Thank you to Dr. Mihai Pop for teaching a biologist string and network algorithms. 
Thanks to Dr. Misaki Takabayshi, my undergraduate research advisor, for introducing me to research and the dark art of PCR.
Thanks to Dr. Jayne Morrow for encouraging me to go back to school and for constant support.
The Corrada-Bravo and Pop labs are filled with brilliant and kind people, with whom it was a pleasure and honor to work.  

This would would not have been possible without the help of many collaborators: 
Dr. Joe Paulson, Justin Wagner, Nidhi Shah, Jayaram Kancherla, and Daniel Hwang helped develop metagenomeFeatures. 
David Catoe, Shan Li, Stephanie Hao, and Dr. Winston Timp helped to generate the assessment dataset.
M. Senthil Kumar and Jacquelyn S Meisel worked on the abundance and beta-diversity assessment chapters respectively. 
I also want to thank Dr. Prachi Kulkarni for editing this dissertation.  

I want to thank my parents for encouraging curiosity, letting me find my own way through life, and for providing me with love and music.
I also want to thank my Grandfather, who is a role model and a constant source of inspiration. 

Most of all I want to thank my wife, Erica, and son Charlie, for loving me through the hard times.
Thanks to Charlie for letting me read microbiome papers to you and for listening to me as I worked though roadblocks in my research.
More importantly, for having an infectious laugh that made it easy to take a break and forget about work for a little while.
